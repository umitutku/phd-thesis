\chapter{Cold Radon Emanation}
\label{chap:chap7}

The use of ultra-radiopure materials is detrimental to the performance of a rare-event search experiment in its capacity to identify, characterise and assign a signal confidence to such rare searches. Furthermore, any hint of a signal in dark matter, neutrinoless double beta decay, or other such rare-searches are modeled against the expected background model, which must be accurately characterised. Radon emanation has emerged to be the most important background limiting the science capability of direct detection G2 experiments such as LZ and XENONnT, despite the screening efforts to date. Future experiments, such as a planned LXe G3 dark matter experiment, will need to control radon backgrounds by a further order of magnitude achieve a sub-dominant radon level, while opening up avenues for new rare searches. Although screening facilities for radon emanation exist, none of these are capable at assessing radon emanation at cryogenic temperatures, important to experiments such as LZ or future LXe G3, in building reliable and accurate background models. The RAL cold radon emanation facility, detailed in this chapter, is an attempt to study radon emanation at 




All rare event search experiments must be constructed from ultra-pure materials following comprehensive radio-assay and screening programmes. Furthermore, expected background levels must be accurately characterised in experimental background models; it is against such high precision models that any hint of signal in dark matter or neutrino-less double beta experiments will be evaluated and signal confidence assigned. Radon emanation has emerged as the most important background limiting science capability in leading experiments such as LZ, where is it already suppressed to unprecedented levels. Future experiments, such as a planned LXe G3 dark matter experiment, will need to control radon backgrounds by a further order of magnitude. Facilities that provide high sensitivity radon emanation assay are rare, with only a handful across the world. None of these, however, are capable of assessing radon emanation at cryogenic temperatures important to experiments such as LZ or future LXe G3, given expected reduction of radon diffusion at low temperatures in many materials. The RAL cold radon emanation facility has been designed with low-temperature capability. This allows, for the first time, systematic measurements of emanation via recoil and diffusion components as a function of temperature for all Xe-wetted materials. The facility also allows testing of novel radon diffusion suppression techniques, from physical barriers using novel materials to chemical treatment of surfaces.



The background from radon emanation for the WIMP search RIO is dominated by the ground-state to ground-state or \textit{naked} \beta-emission from the \PbTOF{} progeny of the \RnTTT{} sub-chain, as it decays to \BiTOF. This results in a uniform ER background with a \beta-spectrum of up to 1019 keV. Similarly, the background from \RnTTZ{} is from the \textit{naked} \beta-emission from \PbTOT{}, as it decays into \BiTOT{} with a \beta-spectrum of up to 569.9 keV. The remaining decays from the sub-radon chain are either too high in energy---in the case of \alpha-decays---or decay with a subsequent particle, i.e. a \gray{} or an \alpha particle, and hence can be vetoed via coincidence tagging, as such is the case for the \beta-decay of \BiTOF{}, which is subsequently followed by the \alpha-decay of \PbTOF{} with a half-life of $\tau{} = 162.3 \; \MathText{\micro{}s}$ \cite{radiogenic_muon_lux,Araujo:2011as}. The details of the decay of each isotope in the radon and thoron sub-chain are summarised in table (\ref{tab:radon_decay_chains}).


%%------------------------------$$
\section{Motivation}
\label{sec:motivation}
%%------------------------------$$



%%------------------------------$$
\section{Design}
\label{sec:motivation}
%%------------------------------$$



All rare event search experiments must be constructed from ultra-pure materials following comprehensive radio-assay and screening programmes. Furthermore, expected background levels must be accurately characterised in experimental background models; it is against such high precision models that any hint of signal in dark matter or neutrino-less double beta experiments will be evaluated and signal confidence assigned. Radon emanation has emerged as the most important background limiting science capability in leading experiments such as LZ, where is it already suppressed to unprecedented levels. Future experiments, such as a planned LXe G3 dark matter experiment, will need to control radon backgrounds by a further order of magnitude. Facilities that provide high sensitivity radon emanation assay are rare, with only a handful across the world. None of these, however, are capable of assessing radon emanation at cryogenic temperatures important to experiments such as LZ or future LXe G3, given expected reduction of radon diffusion at low temperatures in many materials. The RAL cold radon emanation facility has been designed with low-temperature capability. This allows, for the first time, systematic measurements of emanation via recoil and diffusion components as a function of temperature for all Xe-wetted materials. The facility also allows testing of novel radon diffusion suppression techniques, from physical barriers using novel materials to chemical treatment of surfaces.
 
The PhD work would focus on developing an understanding of radon transport and its temperature dependence through materials and subsequent recoil and diffusive emanation in a range between 77K to room temperature and above. The PhD would also include measurements of novel barrier materials, chemical treatment of surfaces, and radon filtering techniques. Finally, the PhD would assess potential for fast in-line radon removal using low-temperature vacuum swing absorption. This PhD work will generate the research required for the future of both direct dark matter and underground neutrinoless double beta decay experiments, otherwise limited by radon. The PhD student will have access to the LZ experimental data for testing and validation of radon transport models at xenon temperature. The research has immediate benefit to experiments such as LZ, informing the background model and specifically radon backgrounds. The development of the cold radon emanation facility and the research proposed, supporting high priority UK science in astroparticle physics​, is timely and necessary for the future of rare event searches.